\documentclass[crop=false, class=book]{standalone}

\usepackage{graphicx}
\usepackage[italian]{varioref}
\usepackage{copyrightbox}

\begin{document}

	\chapter{HitTest}
	
	Un \textit{HitTest} è il risultato che viene restituito quando viene toccato un determinato oggetto \verb|Trackable|.
	Ogni risultato è costituito da:
	\begin{itemize}
		\item \textbf{Lunghezza in metri} dall'origine del raggio che può essere ricavata dall'invocazione del metodo \verb|getDistance()|.
		\item \textbf{Posa} (posizione e orientamento) del punto toccato con \verb|getHitPose()|.
		\item \textbf{Istanza Trackable} che contiene la geometria 3d che è stata toccata con \verb|getTrackable()|.
	\end{itemize}
	
	\noindent
	Questo risultato può essere utilizzato per definire un'ancora che permette di fissare la posizione di contenuti virtuali all'interno dello spazio. L'ancora si adatta agli aggiornamenti dell'ambiente circostante e aggiorna gli oggetti legati ad essa come descritto nel capitolo~\vref{cap:anchor} relativo ad Anchor e Trackable.\\
	
	\noindent
	Esistono quattro tipi di risultati che si possono ottenere in una sessione ARCore:
		\begin{itemize}
		\item \textbf{Profondità}: richiede l'attivazione di depth API nella sessione ARCore ed è usato per posizionare oggetti su superfici arbitrarie (non solo su piani).
		\item \textbf{Aereo}: permette di posizionare un oggetto su superfici piane e utilizza la loro geometria per determinare la profondità e l'orientamento del punto individuato.
		\item \textbf{Punto caratteristico}: permette di disporre oggetti in superfici arbitrarie basandosi su caratteristiche visive attorno al punto sul quale l'utente tocca. 
		\item \textbf{Posizionamento istantaneo}: consente di posizionare un oggetto rapidamente in un piano utilizzando la sua geometria completa attorno al punto selezionato. 
	\end{itemize}
	\clearpage
	
	\section{Definizione e gestione di un HitTest}
	E' possibile ricevere un \verb|HitTest| di tipo diverso come descritto dal listing~\vref{lst: hitTest-filter}.
	\begin{center}
		\begin{minipage}{0.95\textwidth}
			\begin{lstlisting}[caption={Filtraggio hitTest in base al tipo.}, label={lst: hitTest-filter}, language=Kotlin]
			// I risultati dell'hit-test sono ordinati per distanza crescente dalla fotocamera.
			val hitResultList =
  				if (usingInstantPlacement) {
    				// Se si usa la modalità Instant Placement, il valore in 
    				// APPROXIMATE DISTANCE METERS determina quanto lontano sarà 
    				// piazzato l'anchor, dal punto di vista della fotocamera.
    				frame.hitTestInstantPlacement(tap.x, tap.y, APPROXIMATE_DISTANCE_METERS)
    				// I risultati dell'Hit-test usando Instant Placement 
    				// avranno un solo risultato di tipo InstantPlacementResult.
  				} else {
    				frame.hitTest(tap)
  				}
  				
			// Il primo hit result di solito è il più rilevante per rispondere agli input dell'utente.
			val firstHitResult = hitResultList.firstOrNull { hit ->
  					val trackable = hit.trackable!!
  					
  					if(trackable is DepthPoint){
  						// Sostituisci con un qualsiasi oggetto Trackable
  						true
  					} else {
  						false
  					}
  				}
  				
			if (firstHitResult != null) {
  				// Utilizza l'hit result. Ad esempio crea un anchor su tale punto di interesse.
  				val anchor = firstHitResult.createAnchor()
  				// Utilizzo dell'anchor...
			}
			\end{lstlisting}
		\end{minipage}
	\end{center}
	
	
	\noindent
	Per definire un hitTest attraverso un raggio \textbf{arbitrario} si può usare il metodo \verb|Frame.hitTest(origin3: Array<float>, originOffset: int, direction3:| \\
	\verb|Array<float>, originOffset: int)| dove i quattro parametri specificano:
	
	\begin{itemize}
		\item \verb|origin3|: array che contiene le 3 coordinate del punto di partenza del raggio.
		\item \verb|originOffset|: offset sommato alle coordinate dell'array di partenza.
		\item \verb|director3|: array che contiene le 3 coordinate del punto di arrivo del raggio.
		\item \verb|directorOffset|: offset sommato alle coordinate dell'array di arrivo.
	\end{itemize}
	
	\noindent
	
	Per creare un anchor sul risultato del tocco viene usato \verb|hitResult.createAnchor()| che restituirà un anchor disposto 		sul \verb|Trackable| sottostante su cui è avvenuto il tocco.
	Nel caso della nostra applicazione il risultato restituito da hitTest nella modalità \emph{Plane Detection} è di tipo 			\verb|Aereo|; il rilevamento di un piano consente di disporre un animale in un punto preciso. Questo evento è stato gestito 	dal metodo \verb|setOnTapArPlaneListener| riportato nell'esempio di codice \vref{lst: Definizione Anchor in Plane 				Detection}.\\
	Oltre all'oggetto hitTest è stato molto importante \verb|hitTestResult| definito nella documentazione di 						\emph{sceneView}. Questo oggetto mantiene tutti gli hitTest che vengono creati quando l'utente tocca lo schermo. 				Inoltre, contiene le informazioni associate al nodo che è stato colpito dal hitTest. Quando l'utente posiziona un'animale 		in 	un piano, viene creato un oggetto \verb|anchorNode| passando al costruttore l'anchor generato dal 							hitTest. Successivamente, viene aggiunto un oggetto \verb|TransformableNode| come nodo figlio di AnchorNode. Questo 			tipo di nodo può essere utilizzato per aggiungere un oggetto \verbe|Node| come figlio, per eseguire operazioni di 				traslazione, selezione, rotazione e scala. L'utilizzo di hitTestResult è stato utile nell'eliminazione degli animali perchè 	ci ha dato la possibilità di ricavare l'oggetto Node associato all'animale che l'utente voleva eliminare. Per rilevare un 		hitTestResult viene invocato il metodo setOnTouchListener(delNode) su TransformableNode dove delNode è un oggetto di tipo 		Node.OnTouchListener.\\	Nell'esempio \vref{lst: delete-node} è riportato il codice dell'eliminazione di un nodo dalla 			scena.
	
	\begin{center}
		\begin{minipage}{0.95\textwidth}
			\begin{lstlisting}[caption={Eliminazione di un nodo dalla scena in Plane Detection}, label={lst: delete-node}, language=Kotlin]
			//Listener per eliminare i nodi
        	val delNode =
            	Node.OnTouchListener { hitTestResult, motionEvent ->
                	if(switchButton.isChecked){
                    	Log.d(TAG, "handleOnTouch")

                    	// Prima chiamata ad ArFragment per gestire TrasformableNode
                    	arFragment.onPeekTouch(hitTestResult, motionEvent)

                    	//La rimozione si verifica con un evento ACTION_UP
                    	if (motionEvent.action == MotionEvent.ACTION_UP) {

                        	if (hitTestResult.node != null && switchButton.isChecked) {

                            	Log.d(TAG, "handleOnTouch hitTestResult.getNode() != null")

                            	//Restituisce il nodo che è stato colpito dal hitTest
                            	val hitNode: Node? = hitTestResult.node

                            	hitNode!!.renderable = null

                            	hitNode.parent = null

                            	//Eliminazione di tutti i figli del nodo
                            	val children =hitNode.children
                            	if(children.isNotEmpty() && children != null){
                                	for (i in 0 until children.size){
                                    children[i].renderable = null
                                	}
                            	}

                            	arFragment.arSceneView.scene.removeChild(hitNode)
                        	}
                    	}
                	}
                true
            }
			\end{lstlisting}
		\end{minipage}
	\end{center}
	
			
	
\end{document}