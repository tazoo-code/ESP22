\documentclass[crop=false, class=book]{standalone}

%impostazioni lingua
\usepackage[T1]{fontenc}
\usepackage[utf8]{inputenc}
\usepackage[english,italian]{babel}

%sistema i margini
\usepackage{geometry}
\geometry{a4paper,top=2.2cm,bottom=2.2cm,left=3cm,right=3cm, heightrounded}

%interlinea 1.5
\usepackage{setspace}
\onehalfspacing

%gestione delle testatine
\usepackage{fancyhdr}
\pagestyle{fancy}
\lhead{}
\chead{}
\rhead{Titolo}
\lfoot{}
\cfoot{\thepage}
\rfoot{}
\renewcommand{\headrulewidth}{0.4pt}

%formattazione titoli paragrafo
\usepackage{titlesec}
\titleformat{\chapter}[block]{\normalfont\huge\bfseries}{\thechapter.}{0.7em}{\huge}

%pacchetti per i riferimenti in bibliografia
\usepackage[autostyle,italian=guillemets]{csquotes}
\usepackage[style=numeric,citestyle=numeric-comp,backend=biber]{biblatex}

%risorsa che contiene la bibliografia
\addbibresource{./../bibliografia.bib}

\usepackage{lipsum}
\usepackage{graphicx}
\usepackage[italian]{varioref}
\usepackage{copyrightbox}

\begin{document}

	\chapter{Motion tracking}
	
		ARCore usa un processo chiamato \emph{Simultaneous localization and mapping (SLAM)} per determinare lo stato del
		dispositivo che si trova all'interno di un ambiente sconosciuto. Questo stato è descritto dalla sua posa (posizione 
		e orientazione) che viene stimata attraverso prestazioni di odometria eccezionali e rilevazione di punti 						caratteristici. Con odometria si intende l'uso di dati ricavati da sensori di movimento che permettono di valutare il 			cambiamento della posizione nel tempo. Nel caso degli smartphone viene utilizzato il sensore IMU che 							rileva misure inerziali come la velocita, accelerazione e posizione. La rilevazione di punti caratteristici è 					l'individuazione di immagini con caratteristiche differenti che consentono al dispositivo di calcolare la sua posizione 		relativa. Questi punti di riferimento insieme alle misurazioni ricavate dai sensori permettono di avere una 					buona stima della posa e di ricavare la rappresentazione di una mappa dell'ambiente circostante. Tuttavia, 						il movimento sequenziale stimato dallo SLAM include un certo margine di errore che si accumula nel tempo causando 				una notevole deviazione dai valori reali. Una soluzione che può essere adottata per risolvere questo 							problema consiste nel considerare come punto di riferimento un luogo visitato in precedenza di cui si sono 						memorizzate le sue caratteristiche. Grazie alle informazioni di questo luogo è possibile minimizzare l'errore nella 			stima della posa.\\
		I contenuti virtuali possono essere renderizzati nella giusta prospettiva allineando la posa della telecamera virtuale 			con quella calcolata da ARCore. Il contenuto virtuale sembra reale perchè è sovrapposto all'immagine ottenura dalla 			fotocamera del dispositivo.
	
\end{document}