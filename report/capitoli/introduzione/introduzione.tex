\documentclass[crop=false, class=book]{standalone}

%impostazioni lingua
\usepackage[T1]{fontenc}
\usepackage[utf8]{inputenc}
\usepackage[english,italian]{babel}

%sistema i margini
\usepackage{geometry}
\geometry{a4paper,top=2.2cm,bottom=2.2cm,left=3cm,right=3cm, heightrounded}

%interlinea 1.5
\usepackage{setspace}
\onehalfspacing


%formattazione titoli paragrafo
\usepackage{titlesec}
\titleformat{\chapter}[block]{\normalfont\huge\bfseries}{\thechapter.}{0.7em}{\huge}

%pacchetti per i riferimenti in bibliografia
\usepackage[autostyle,italian=guillemets]{csquotes}
\usepackage[sorting=none,style=numeric,citestyle=numeric-comp,backend=biber]{biblatex}

%risorsa che contiene la bibliografia
\addbibresource{./../../bibliografia.bib}



\begin{document}
	\chapter{Introduzione}
	\textit{ARCore} è un kit di sviluppo lanciato da Google nel mese di marzo 2018, utilizzabile nella maggior parte degli smartphone con Android Nougat o superiore (API level 24+). Tramite esso è possibile sviluppare applicazioni con funzionalità in realtà aumentata, permettendo all'utente di interagire con l'ambiente che lo circonda.
	In questo documento, dopo una breve panoramica sulla realtà aumentata, verranno analizzate le principali funzionalità e caratteristiche dell'SDK.
	\\
	Nei primi capitoli vengono presentate le tre funzioni fondamentali del framework ARCore, che permettono al dispositivo di integrare contenuti virtuali al mondo reale:
	\begin{itemize}
		\item Il rilevamento del movimento, che consente ad esso di tracciare la propria posizione nel mondo.
		\item La comprensione ambientale, che permette la rilevazione della posizione e della dimensione delle superfici.
		\item La stima della luce, che consente di valutare le condizioni di illuminazione dell'ambiente.
	\end{itemize}
	Nei capitoli successivi sono invece presentate le funzionalità aggiuntive del framework, che consentono di migliorare l'integrazione tra virtuale e reale, come ad esempio il rilevamento della profondità, il posizionamento istantaneo, le API \textit{Augmented Images} e \textit{Augmented Faces} e altre importanti funzioni aggiuntive.
\end{document}