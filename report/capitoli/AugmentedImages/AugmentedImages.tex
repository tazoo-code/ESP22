\documentclass[crop=false, class=book]{standalone}

%impostazioni lingua
\usepackage[T1]{fontenc}
\usepackage[utf8]{inputenc}
\usepackage[english,italian]{babel}

%sistema i margini
\usepackage{geometry}
\geometry{a4paper,top=2.2cm,bottom=2.2cm,left=3cm,right=3cm, heightrounded}

%interlinea 1.5
\usepackage{setspace}
\onehalfspacing

%gestione delle testatine
\usepackage{fancyhdr}
\pagestyle{fancy}
\lhead{}
\chead{}
\rhead{Titolo}
\lfoot{}
\cfoot{\thepage}
\rfoot{}
\renewcommand{\headrulewidth}{0.4pt}

%formattazione titoli paragrafo
\usepackage{titlesec}
\titleformat{\chapter}[block]{\normalfont\huge\bfseries}{\thechapter.}{0.7em}{\huge}

%pacchetti per i riferimenti in bibliografia
\usepackage[autostyle,italian=guillemets]{csquotes}
\usepackage[style=numeric,citestyle=numeric-comp,backend=biber]{biblatex}

%risorsa che contiene la bibliografia
\addbibresource{./bibliografia.bib}

%pacchetto per immagini
\usepackage{graphicx}
\usepackage{subfig}

\usepackage[italian]{varioref}
\usepackage{copyrightbox}
\usepackage{url}
\usepackage{lipsum}




\begin{document}
	\chapter{Augmented Images}
	
	
	Le API Augmented Images in ARCore consentono di rilevare e poi manipolare le immagini presenti nel mondo reale e integrarle con elementi del mondo virtuale.
	
Dopo aver fornito delle immagini, ARCore con un algoritmo che estrae dei punti caratteristici in scala di grigi dall’immagine, ripone tutti questi dettagli in uno o più database.

Durante l’esecuzione, ARCore cerca questi punti nell’ambiente consentendo di rilevare la posizione, l’orientamento e la dimensione dell’immagine nello spazio.

	\section{Capacità}
	ARCore riesce a individuare fino a 20 immagini in contemporanea e non individua due istanze della stessa immagine.
	
Il suo database riesce a contenere fino a 1000 immagini ed inoltre, non c’è limite al numero di database che possono essere caricati. L’unico vincolo è che si può attivare ed usare un singolo database alla volta.

Quando si aggiunge un’immagine è possibile indicargli la sua grandezza nell’ambiente così da velocizzare il processo di tracciamento.

Se non viene fornita la dimensione dell’immagine, ARCore stimerà la sua grandezza e la migliorerà durante l’esecuzione. Se viene fornita la sua dimensione ARCore ignorerà la discrepanza tra la dimensione attuale e quella segnalata e scalerà tutto secondo la dimensione dell’immagine dichiarata.

ARCore è in grado di rilevare e tracciare sia immagini fisse, come ad esempio un poster nel muro, oppure immagini in movimento, come ad esempio un utente che ha l’immagine nella sua mano.

Una volta che l’immagine viene rilevata ARCore continua ad ottenere informazioni ed a rifinire la sua stima dell’immagine nell’ambiente (FULL\_TRACKING).

Se l’immagine si muove fuori dall’inquadratura ARCore continua a stimare la sua posizione assumendo che l’immagine sia statica e non si muova nell’ambiente (LAST\_KNOWN\_POSE).

	
\section{Requisiti}

L’immagine deve:

•	Essere visibile almeno al 25\% nell'inquadratura

•	Essere piatta e non accartocciata

•	Essere libera da ogni oggetto tra la fotocamera e l'immagine. Non deve essere oscurata, vista troppo angolata oppure vista quando la fotocamera si muove troppo velocemente

\section{Utilizzo della CPU e considerazioni sulle performance}

In base alle funzionalità di ARCore abilitate, la batteria potrebbe scaricarsi più velocemente e può aumentare l’utilizzo della CPU. E’ consigliato disabilitare le funzionalità non utilizzate per preservare la batteria e rendere più leggero il carico della CPU.

\section{Creazione del database}

Esistono due metodi per creare un database di immagini che verranno successivamente usate da ARCore.

Il primo metodo è importare un database creato in precedenza con il tool arcoreimg.
Per importarlo si usano le seguenti righe di codice:

\begin{center}
	\begin{minipage}{0.95\textwidth}
	
	Codice...

val imageDatabase = this.assets.open("example.imgdb").use {

  AugmentedImageDatabase.deserialize(session, it)
  
}
	
	\end{minipage}
\end{center}

Il secondo metodo consiste nel crearlo a runtime partendo con delle immagini contenuti nella cartella assets dell’applicazione. Lo si può fare per esempio con le seguenti righe di codice:

\begin{center}
	\begin{minipage}{0.95\textwidth}
	
	Codice...

val imageDatabase = AugmentedImageDatabase(session)

val bitmap = assets.open("dog.jpg").use { BitmapFactory.decodeStream(it) }

// If the physical size of the image is not known, use addImage(String, Bitmap) instead, at the

// expense of an increased image detection time.

val imageWidthInMeters = 0.10f // 10 cm

val dogIndex = imageDatabase.addImage("dog", bitmap, imageWidthInMeters)

  
\}
	
	\end{minipage}
\end{center}

\section{Come abilitare il tracciamento delle immagini}
Per abilitare il tracciamento delle immagini bisogna modificare la configurazione della sessione per impostare il database da utilizzare. Lo si può fare così:

\begin{center}
	\begin{minipage}{0.95\textwidth}
	
	val config = Config(session)
	
config.augmentedImageDatabase = imageDatabase

session.configure(config)

	
	
	\end{minipage}
\end{center}



\end{document}
