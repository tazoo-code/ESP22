\documentclass[a4paper,12pt,final, openany, titlepage, twoside]{book}

%permette di aggiungere altri pacchetti dai file inclusi
\usepackage[subpreambles]{standalone} 

%impostazioni lingua
\usepackage[T1]{fontenc}
\usepackage[utf8]{inputenc}
\usepackage[english,italian]{babel}

%sistema i margini
\usepackage{geometry}
\geometry{a4paper,top=2.2cm,bottom=2.2cm,left=3cm,right=3cm, heightrounded}

%interlinea 1.5
\usepackage{setspace}
\onehalfspacing

%gestione delle testatine
\usepackage{fancyhdr}
\pagestyle{fancy}
\lhead{}
\chead{}
\rhead{Titolo}
\lfoot{}
\cfoot{\thepage}
\rfoot{}
\renewcommand{\headrulewidth}{0.4pt}

%formattazione titoli paragrafo
\usepackage{titlesec}
\titleformat{\chapter}[block]{\normalfont\huge\bfseries}{\thechapter.}{0.7em}{\huge}


\usepackage[autostyle,italian=guillemets]{csquotes}
\usepackage[sorting=none,style=numeric,citestyle=numeric-comp,backend=biber]{biblatex}

\addbibresource{bibliografia.bib}

%collegamenti ipertestuali
%\usepackage{hyperref}

%genera testo casuale
\usepackage{lipsum}

\begin{document}
	
	
	\frontmatter
	\documentclass[crop=false]{standalone}

\usepackage{frontespizio}

\begin{document}
	\begin{frontespizio}
		\Preambolo{\renewcommand{\fronttitlefont}{%
				\fontsize{22}{22}\bfseries}}
		
		\Universita{Padova}
		\Logo[3cm]{./resources/images/loghi.jpg}
		\Dipartimento{Ingegneria dell'Informazione}
		\Corso[Laurea]{Ingegneria Informatica}
		\Titolo{Google ARCore}
		%aggiunge più candidati
		\NCandidati{}
		%elimina il carattere ":"
		\Punteggiatura{}
		\Preambolo{\renewcommand{\frontcandidatesep}{0.5cm}}
		\Candidato[1216433]{Marco Vettore}
		\Candidato[1227153]{Alberto Varini}
		\Candidato[1219603]{Mattia Tamiazzo}
		\Rientro{1cm}
		\Annoaccademico{2021-2022}
		\Margini{1.5cm}{1cm}{1.5cm}{1cm}
	\end{frontespizio}
\end{document}

	\documentclass[crop=false]{standalone}

\begin{document}
	\tableofcontents
\end{document}
	
	
	\mainmatter
	\documentclass[crop=false, class=book]{standalone}

%impostazioni lingua
\usepackage[T1]{fontenc}
\usepackage[utf8]{inputenc}
\usepackage[english,italian]{babel}

%sistema i margini
\usepackage{geometry}
\geometry{a4paper,top=2.2cm,bottom=2.2cm,left=3cm,right=3cm, heightrounded}

%interlinea 1.5
\usepackage{setspace}
\onehalfspacing

%gestione delle testatine
\usepackage{fancyhdr}
\pagestyle{fancy}
\lhead{}
\chead{}
\rhead{Titolo}
\lfoot{}
\cfoot{\thepage}
\rfoot{}
\renewcommand{\headrulewidth}{0.4pt}

%formattazione titoli paragrafo
\usepackage{titlesec}
\titleformat{\chapter}[block]{\normalfont\huge\bfseries}{\thechapter.}{0.7em}{\huge}

%pacchetti per i riferimenti in bibliografia
\usepackage[autostyle,italian=guillemets]{csquotes}
\usepackage[sorting=none,style=numeric,citestyle=numeric-comp,backend=biber]{biblatex}

%risorsa che contiene la bibliografia
\addbibresource{./../bibliografia.bib}



\begin{document}
	\chapter{Introduzione}
	La Realtà Aumentata (Augmented Reality - \emph{AR}) è una tecnologia che permette di compiere esperienze interattive, in cui l'ambiente reale viene arricchito da contenuti virtuali. Similmente alla realtà virtuale, vengono creati elementi grafici sintetici con cui l'utente può interagire attraverso i sensi. Tuttavia, come spiegato in \cite{bimber2005spatial}, nell'AR l'ambiente reale gioca un ruolo fondamentale: lo scopo della realtà aumentata è proprio cercare di collegare il mondo reale con quello virtuale.
	\\
	L'AR viene definita in \cite{azuma1997survey} come un sistema che incorpora tre caratteristiche principali: la combinazione tra reale e virtuale, l'interazione real-time e la rappresentazione 3D. 
	
	\section{Storia}
	Il primo rudimentale sistema di realtà aumentata è stato creato da Ivan Sutherland \cite{sutherland1968head} nel 1968. Esso era composto da un display ottico trasparente che veniva montato sulla testa e che poteva mostrare semplici immagini in tempo reale. Nel 1993 George Fitzmaurice ha creato \textit{Chameleon} \cite{fitzmaurice1993situated}, un dispositivo che tramite un piccolo schermo collegato a una videocamera poteva essere orientato per esplorare uno spazio virtuale 3D. Simile al prototipo di Fitzmaurice, nel 1995 Jun Rekimoto e Katashi Nagao creano \textit{NaviCam} \cite{rekimoto1995world}, che prendendo in input un flusso video poteva riconoscere in real-time dei marcatori colorati e sovrapporre al video delle informazioni testuali. 
	\\
	Dal 2000 vengono creati altri primi sistemi di realtà aumentata, con applicazioni soprattutto a giochi interattivi, come ad esempio l'estensione \textit{ARQuake} \cite{thomas2000arquake} o \textit{Human Pacman} \cite{cheok2003human}, ancora vincolati alle scarse prestazioni dei dispositivi mobili. Solo con l'aggiunta ai cellulari della fotocamera, e poi di schermi touch, vengono quindi create le prime applicazioni commerciali in grado di sfruttare le potenzialità della realtà aumentata. Ne sono esempi \textit{AR Tennis} \cite{henrysson2006ARtennis}, primo gioco in AR collaborativo per cellulare, e \textit{ARhrrrr!}, primo gioco mobile in realtà aumentata con contenuti grafici di alta qualità. Vengono quindi sviluppate le prime librerie software per la realtà aumentata, come \textit{ARToolKit}, implementata prima in linguaggio C e poi in C++ nelle versioni più recenti, \textit{OpenCV}, che possiede anche funzionalità per l'AR, e dal 2018 la libreria per Android \textit{ARCore} di Google.
	
	\section{Applicazioni}	
	Le applicazioni della realtà aumentata possono riguardare diversi ambiti, ai quali questa tecnologia può apportare benefici economici o qualitativi, oppure creare servizi innovativi \cite{carmigniani2011augmented}. 
	\\
	In particolare, nel corso degli anni la tecnologia AR è stata usata per scopi pubblicitari e commerciali, quali la prova di capi d'abbigliamento senza doverli indossare o l'integrazione al marketing cartaceo di video promozionali tramite riconoscimento delle immagini, o per l'intrattenimento, come nello sviluppo di videogiochi. Trova inoltre applicazioni nella produzione industriale, in cui vengono sovrapposte all'area di lavoro istruzioni virtuali, nell'ambito militare, come l'addestramento al volo dei piloti, o per la formazione e la pratica sanitaria.
\end{document}
	\documentclass[crop=false]{standalone}

\begin{document}
	\chapter{Capitolo uno}
		\lipsum[1-5]
	\section{Paragrafo uno}
		\lipsum[1-5]
	\subsection{Sottoparagrafo uno}
		\lipsum[1-5]
\end{document}
	\section{d}
	\lipsum[1-10] 
	\backmatter
	\documentclass[crop=false]{standalone}


\usepackage[autostyle,italian=guillemets]{csquotes}
\usepackage[style=numeric,citestyle=numeric-comp,backend=biber]{biblatex}

\addbibresource{bibliografia.bib}

\begin{document}
	\printbibliography[heading=bibintoc]
\end{document}
	
	
\end{document}
