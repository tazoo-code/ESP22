\documentclass[a4paper,12pt,final, openany, titlepage, twoside]{book}

%permette di aggiungere altri pacchetti dai file inclusi
\usepackage[subpreambles]{standalone} 

%impostazioni lingua
\usepackage[T1]{fontenc}
\usepackage[utf8]{inputenc}
\usepackage[english,italian]{babel}

%sistema i margini
\usepackage{geometry}
\geometry{a4paper,top=2.2cm,bottom=2.2cm,left=3cm,right=3cm, heightrounded}

%interlinea 1.5
\usepackage{setspace}
\onehalfspacing

%gestione delle testatine
\usepackage{fancyhdr}
\pagestyle{fancy}
\lhead{}
\chead{}
\rhead{Titolo}
\lfoot{}
\cfoot{\thepage}
\rfoot{}
\renewcommand{\headrulewidth}{0.4pt}

%formattazione titoli paragrafo
\usepackage{titlesec}
\titleformat{\chapter}[block]{\normalfont\huge\bfseries}{\thechapter.}{0.7em}{\huge}


\usepackage[autostyle,italian=guillemets]{csquotes}
\usepackage[style=numeric,citestyle=numeric-comp,backend=biber]{biblatex}

\addbibresource{bibliografia.bib}

%collegamenti ipertestuali
%\usepackage{hyperref}

%genera testo casuale
\usepackage{lipsum}

\begin{document}
	
	
	\frontmatter
	\documentclass[crop=false]{standalone}

\usepackage{frontespizio}

\begin{document}
	\begin{frontespizio}
		\Preambolo{\renewcommand{\fronttitlefont}{%
				\fontsize{22}{22}\bfseries}}
		
		\Universita{Padova}
		\Logo[3cm]{./resources/images/loghi.jpg}
		\Dipartimento{Ingegneria dell'Informazione}
		\Corso[Laurea]{Ingegneria Informatica}
		\Titolo{Google ARCore}
		%aggiunge più candidati
		\NCandidati{}
		%elimina il carattere ":"
		\Punteggiatura{}
		\Preambolo{\renewcommand{\frontcandidatesep}{0.5cm}}
		\Candidato[1216433]{Marco Vettore}
		\Candidato[1227153]{Alberto Varini}
		\Candidato[1219603]{Mattia Tamiazzo}
		\Rientro{1cm}
		\Annoaccademico{2021-2022}
		\Margini{1.5cm}{1cm}{1.5cm}{1cm}
	\end{frontespizio}
\end{document}

	\documentclass[crop=false]{standalone}

\begin{document}
	\tableofcontents
\end{document}
	
	
	\mainmatter
	\documentclass[crop=false]{standalone}

\begin{document}
	\chapter{Capitolo uno}
		\lipsum[1-5]
	\section{Paragrafo uno}
		\lipsum[1-5]
	\subsection{Sottoparagrafo uno}
		\lipsum[1-5]
\end{document}
	\section{d}
	\lipsum[1-10] Come si può notare in \cite{bimber2005spatial}
	\backmatter
	\documentclass[crop=false]{standalone}


\usepackage[autostyle,italian=guillemets]{csquotes}
\usepackage[style=numeric,citestyle=numeric-comp,backend=biber]{biblatex}

\addbibresource{bibliografia.bib}

\begin{document}
	\printbibliography[heading=bibintoc]
\end{document}
	
	
\end{document}
